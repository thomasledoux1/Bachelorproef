%%=============================================================================
%% Resultaten
%%=============================================================================

\chapter{Resultaten}
\label{ch:resultaten}

\section{Vergelijking verschillende technieken}
\label{sec:technieken}

%% TODO:
%% Uit je probleemstelling moet duidelijk zijn dat je onderzoek een meerwaarde
%% heeft voor een concrete doelgroep (bv. een bedrijf).
%%
%% Wees zo concreet mogelijk bij het formuleren van je
%% onderzoeksvra(a)g(en). Een onderzoeksvraag is trouwens iets waar nog
%% niemand op dit moment een antwoord heeft (voor zover je kan nagaan).
Verschillende van de voorgestelde technieken worden reeds automatisch toegepast door Android Studio bij het compileren van de applicatie. Zo zit Proguard ingebouwd in Android Studio, worden ongebruikte resources verwijderd door de ingebouwde lint tool en kunnen enumeraties automatisch omgezet worden naar integers. Hierdoor zal het handmatig uitvoeren van deze technieken geen extra effect teweeg brengen. Daarnaast zal het effect van de technieken afhankelijk zijn van het app-profiel. 
Legende technieken : 
\begin{enumerate}
	\item Techniek0 = Geen compressietechniek toegepast
	\item Techniek1 = Verwijder ongebruikte resources
	\item Techniek2 = Comprimeer afbeeldingen
	\item Techniek3 = Gebruik WebP-formaat
	\item Techniek4 = Verklein grootte native binaries
\end{enumerate}
\subsection{Foto-app}
Bij de eerste applicatie die getest werd, een foto-app die veel mediabestanden bevat, had vooral het gebruik van het WebP-formaat voor alle afbeeldingen een grote invloed op de grootte van de APK. De andere technieken brachten op dit vlak geen verandering. Op vlak van het geheugengebruik van de applicatie tijdens runtime kon een minieme verandering opgemerkt worden na het toevoegen van de techniek ``Verklein resource gebruik van libraries``. Na de omzetting van de afbeeldingen naar het WebP-formaat verhoogde het CPU-gebruik met 9 procent. Dit wordt veroorzaakt doordat het genereren van de WebP-afbeeldingen meer CPU-kracht vergt.
\begin{figure}[H]
	\centering
	\caption{\textit{Resultaten verschillende compressietechnieken op APK-grootte foto-app}}
	\includegraphics[width=10cm, height=10cm, keepaspectratio]{img/Rplot01}\\[.5cm]
	
\end{figure}

\subsection{Fitness-app}
\begin{figure}[H]
	\centering
	\caption{\textit{Resultaten verschillende compressietechnieken op APK-grootte fitness-app}}
	\includegraphics[width=10cm, height=10cm, keepaspectratio]{img/Rplot02}\\[.5cm]
	
\end{figure}
\subsection{CPU-intensieve app}
\begin{figure}[H]
	\centering
	\caption{\textit{Resultaten verschillende compressietechnieken op APK-grootte CPU-intensieve app}}
	\includegraphics[width=10cm, height=10cm, keepaspectratio]{img/Rplot02}\\[.5cm]
	
\end{figure}

\section{Invloed compressie op prestaties apps}
\label{sec:invloedcompressie}
\subsection{Foto-app}
\subsubsection{Geheugengebruik}
\begin{figure}[H]
	\centering
	\caption{\textit{Resultaten verschillende compressietechnieken op geheugengebruik foto-app}}
	\includegraphics[width=10cm, height=10cm, keepaspectratio]{img/Rplot03}\\[.5cm]
	
\end{figure}
\subsubsection{CPU-gebruik}
\begin{figure}[H]
	\centering
	\caption{\textit{Resultaten verschillende compressietechnieken op CPU-gebruik foto-app}}
	\includegraphics[width=10cm, height=10cm, keepaspectratio]{img/app1cpu}\\[.5cm]
	
\end{figure}
\subsection{Fitness-app}
\subsubsection{Geheugengebruik}
\begin{figure}[H]
	\centering
	\caption{\textit{Resultaten verschillende compressietechnieken op geheugengebruik fitness-app}}
	\includegraphics[width=10cm, height=10cm, keepaspectratio]{img/app2geheugen}\\[.5cm]
	
\end{figure}
\subsubsection{CPU-gebruik}
\begin{figure}[H]
	\centering
	\caption{\textit{Resultaten verschillende compressietechnieken op CPU-gebruik fitness-app}}
	\includegraphics[width=10cm, height=10cm, keepaspectratio]{img/app2cpu}\\[.5cm]
	
\end{figure}
\subsection{CPU-intensieve app}
\subsubsection{Geheugengebruik}
\begin{figure}[H]
	\centering
	\caption{\textit{Resultaten verschillende compressietechnieken op geheugengebruik CPU-intensieve app}}
	\includegraphics[width=10cm, height=10cm, keepaspectratio]{img/app3geheugen}\\[.5cm]
	
\end{figure}
\subsubsection{CPU-gebruik}
\begin{figure}[H]
	\centering
	\caption{\textit{Resultaten verschillende compressietechnieken op CPU-gebruik CPU-intensieve app}}
	\includegraphics[width=10cm, height=10cm, keepaspectratio]{img/app3cpu}\\[.5cm]
	
\end{figure}

 \section{Testresultaten framework Brian Pinsard}
\label{sec:invloedcompressie}