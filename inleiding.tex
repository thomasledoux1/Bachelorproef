%%=============================================================================
%% Inleiding
%%=============================================================================

\chapter{Inleiding}
\label{ch:inleiding}

\section{Probleemstelling en Onderzoeksvragen}
\label{sec:onderzoeksvragen}

%% TODO:
%% Uit je probleemstelling moet duidelijk zijn dat je onderzoek een meerwaarde
%% heeft voor een concrete doelgroep (bv. een bedrijf).
%%
%% Wees zo concreet mogelijk bij het formuleren van je
%% onderzoeksvra(a)g(en). Een onderzoeksvraag is trouwens iets waar nog
%% niemand op dit moment een antwoord heeft (voor zover je kan nagaan).
We beschrijven en vergelijken technieken om de grootte van apps voor Android Wear te beperken. De voorbije jaren hebben wearable devices steeds meer aan belang gewonnen in het dagelijks leven. Volgens een recent onderzoek werd het aantal verkochte wearable devices wereldwijd geschat op 20.1 miljoen, waarvan 22.9 procent Android Wear als operating system gebruikt. IDC voorspelt tegen 2020 dat er 54.6 miljoen wearable devices verkocht zullen worden, waarvan 41.8 procent Android Wear als operating system zal gebruiken. \autocite{IDC}. Elk noemenswaardig bedrijf heeft tegenwoordig een app laten ontwikkelen die ook op wearable devices kan gebruikt worden om zoveel mogelijk van deze mobiele gebruikers te bereiken. Hiermee gaat echter gepaard dat gebruikers meer en meer apps willen installeren op hun wearables, terwijl deze slechts over een beperkte schijfruimte beschikken. Daarom moeten de ontwikkelaars van deze mobiele applicaties ervoor zorgen dat de grootte van hun applicaties zoveel mogelijk beperkt wordt, zodat mobiele toestellen niet zonder vrije schijfruimte komen te staan. We kiezen voor Android Wear omdat dit een open-source platform is. De belangrijkste pijlers van open-source zijn dat open-source software vrij kan gedistribueerd worden en dat de source code beschikbaar wordt gesteld. \autocite{OpenSourceInitiative}. Dit vergemakkelijkt het onderzoek doordat er in de broncode kan gekeken worden welke technieken precies gebruikt worden en op welke manier deze geïmplementeerd worden. Daardoor kunnen er voor het testen van de verschillende technieken open-source apps gebruikt worden, die al verder uitgewerkt zijn dan de app die tijdens dit onderzoek gecreëerd werd. Android Wear toestellen beschikken vaak maar over een beperkte opslagruimte, waardoor elke megabyte vrije ruimte telt. De eindgebruikers van de apps merken hier dus het meest directe effect van, zij zullen kunnen profiteren van de vrijgekomen ruimte door de compressie van de apps. Maar ook voor bedrijven en hun app-ontwikkelaars is dit belangrijk: wanneer hun Android Wear app niet te veel schrijfruimte inneemt op het toestel van de eindgebruikers, zullen zij minder snel geneigd zijn deze app te verwijderen. Wanneer ze de app wel zouden verwijderen ontloopt het bedrijf belangrijke inkomsten uit in-app purchases en eventuele reclame-inkomsten. Het vergelijken van de verschillende technieken zal gebeuren op verschillende vlakken: zijn er veel media-bestanden aanwezig in de applicatie? Is er veel code aanwezig in de applicatie? Zijn er meerdere APK's voorzien?

\section{Opzet van deze bachelorproef}
\label{sec:opzet-bachelorproef}

%% TODO: Het is gebruikelijk aan het einde van de inleiding een overzicht te
%% geven van de opbouw van de rest van de tekst. Deze sectie bevat al een aanzet
%% die je kan aanvullen/aanpassen in functie van je eigen tekst.

%%De rest van deze bachelorproef is als volgt opgebouwd:

%%In Hoofdstuk~\ref{ch:methodologie} wordt de methodologie toegelicht en worden de gebruikte onderzoekstechnieken besproken om een antwoord te kunnen formuleren op de onderzoeksvragen.

%% TODO: Vul hier aan voor je eigen hoofstukken, één of twee zinnen per hoofdstuk

%%In Hoofdstuk~\ref{ch:conclusie}, tenslotte, wordt de conclusie gegeven en een antwoord geformuleerd op de onderzoeksvragen. Daarbij wordt ook een aanzet gegeven voor toekomstig onderzoek binnen dit domein.
Om dit onderzoek te ondersteunen zal een proof-of-concept app gecreëerd worden voor Android Wear. Op deze app zullen dan de verschillende technieken toegepast worden en zal het effect van deze technieken vergeleken en besproken worden. In de literatuurstudie bespreken we wat Android Wear precies inhoudt en bespreken we de verschillende compressietechnieken. Binnen de literatuurstudie kijken we ook welke verschillende profielen van apps er onderzocht moeten worden. Zo zullen er apps zijn die veelvuldig gebruik maken van media-bestanden, bijvoorbeeld foto-apps, waarbij compressie dus vooral zal gericht zijn op de foto's. Anderzijds zullen er ook apps zijn die gebruikt worden tijdens het sporten, waarbij GPS-tracking een grote rol zal spelen. Als laatste categorie zullen apps onderzocht worden die CPU-intensieve taken zullen uitvoeren, bijvoorbeeld een app die muziek afspeelt en op het tempo ervan live visuele effecten genereert. Er zal ook gekeken worden of er al dan niet een verband is tussen de compressie van de apps en de prestatie ervan op vlak van geheugengebruik en CPU-gebruik. Nadien zullen de resultaten van het onderzoek besproken worden en vergeleken worden. Tot slot volgt een samenvattende conclusie van het onderzoek. 


