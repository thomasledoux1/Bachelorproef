%%=============================================================================
%% Inleiding
%%=============================================================================

\chapter{Inleiding}
\label{ch:inleiding}

\section{Probleemstelling en Onderzoeksvragen}
\label{sec:onderzoeksvragen}

%% TODO:
%% Uit je probleemstelling moet duidelijk zijn dat je onderzoek een meerwaarde
%% heeft voor een concrete doelgroep (bv. een bedrijf).
%%
%% Wees zo concreet mogelijk bij het formuleren van je
%% onderzoeksvra(a)g(en). Een onderzoeksvraag is trouwens iets waar nog
%% niemand op dit moment een antwoord heeft (voor zover je kan nagaan).
In deze bachelorproef worden verschillende technieken beschreven en vergeleken om de grootte van apps voor Android Wear te beperken. Er werd voor Android Wear gekozen omdat dit een open-source platform is. De belangrijkste pijlers van open-source zijn dat open-source software vrij kan gedistribueerd worden en dat de source code beschikbaar wordt gesteld volgens \textcite{Open Source Initiative} . Dit vergemakkelijkt het onderzoek doordat er in de broncode kan gekeken worden welke technieken precies gebruikt worden, en op welke manier deze geïmplementeerd worden. Android Wear toestellen beschikken vaak maar over een beperkte opslagruimte, waardoor elke megabyte vrije ruimte telt. De eindgebruikers van de apps hebben hier dus het meest directe effect van, zij zullen kunnen profiteren van de vrijgekomen ruimte door de compressie van de apps. Maar ook voor bedrijven en hun app-ontwikkelaars is dit belangrijk, wanneer hun Android Wear app niet te veel schrijfruimte inneemt op het toestel van de eindgebruikers, zullen zij minder snel geneigd zijn deze app te verwijderen. Wanneer ze de app wel zouden verwijderen ontloopt het bedrijf belangrijke inkomsten uit in-app purchases en eventuele reclame-inkomsten. Het vergelijken van de verschillende technieken zal gebeuren op verschillende vlakken : Zijn er veel media-bestanden aanwezig in de applicatie? Is er veel code aanwezig in de applicatie? Zijn er meerdere APK's voorzien?

\section{Opzet van deze bachelorproef}
\label{sec:opzet-bachelorproef}

%% TODO: Het is gebruikelijk aan het einde van de inleiding een overzicht te
%% geven van de opbouw van de rest van de tekst. Deze sectie bevat al een aanzet
%% die je kan aanvullen/aanpassen in functie van je eigen tekst.

%%De rest van deze bachelorproef is als volgt opgebouwd:

%%In Hoofdstuk~\ref{ch:methodologie} wordt de methodologie toegelicht en worden de gebruikte onderzoekstechnieken besproken om een antwoord te kunnen formuleren op de onderzoeksvragen.

%% TODO: Vul hier aan voor je eigen hoofstukken, één of twee zinnen per hoofdstuk

%%In Hoofdstuk~\ref{ch:conclusie}, tenslotte, wordt de conclusie gegeven en een antwoord geformuleerd op de onderzoeksvragen. Daarbij wordt ook een aanzet gegeven voor toekomstig onderzoek binnen dit domein.
Om dit onderzoek te ondersteunen zal een proof-of-concept app gecreërd worden voor Android Wear. Op deze app zullen dan de verschillende technieken toegepast worden, en zal het effect van deze technieken vergeleken en besproken worden. Er wordt ook een literatuurstudie uitgevoerd waarin zal besproken worden wat Android Wear precies inhoudt, en waar de verschillende compressietechnieken besproken worden. Er zal ook gekeken worden of er al dan niet een verband is tussen de compressie van de apps en de prestatie ervan op vlak van geheugengebruik. Nadien zullen de resultaten van het onderzoek besproken worden. Tot slot volgt een samenvattende conclusie van het onderzoek. 


