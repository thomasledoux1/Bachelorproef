%%=============================================================================
%% Samenvatting
%%=============================================================================

%% TODO: De "abstract" of samenvatting is een kernachtige (~ 1 blz. voor een
%% thesis) synthese van het document.
%%
%% Deze aspecten moeten zeker aan bod komen:
%% - Context: waarom is dit werk belangrijk?
%% - Nood: waarom moest dit onderzocht worden?
%% - Taak: wat heb je precies gedaan?
%% - Object: wat staat in dit document geschreven?
%% - Resultaat: wat was het resultaat?
%% - Conclusie: wat is/zijn de belangrijkste conclusie(s)?
%% - Perspectief: blijven er nog vragen open die in de toekomst nog kunnen
%%    onderzocht worden? Wat is een mogelijk vervolg voor jouw onderzoek?
%%
%% LET OP! Een samenvatting is GEEN voorwoord!

%%---------- Nederlandse samenvatting -----------------------------------------
%%
%% TODO: Als je je bachelorproef in het Engels schrijft, moet je eerst een
%% Nederlandse samenvatting invoegen. Haal daarvoor onderstaande code uit
%% commentaar.
%% Wie zijn bachelorproef in het Nederlands schrijft, kan dit negeren en heel
%% deze sectie verwijderen.

\IfLanguageName{english}{%
\selectlanguage{dutch}
\chapter*{Samenvatting}
\lipsum[1-4]
\selectlanguage{english}
}{}

%%---------- Samenvatting -----------------------------------------------------
%%
%% De samenvatting in de hoofdtaal van het document

\chapter*{\IfLanguageName{dutch}{Samenvatting}{Abstract}}

Ontwikkelaars van Android Wear apps proberen de grootte van hun apps te verkleinen. Dit is noodzakelijk omdat Android Wear toestellen vaak over een een kleine interne opslagruimte beschikken. Om de apps in grootte te verkleinen zijn verschillende compressietechnieken voorhanden. Wij hebben deze technieken met elkaar vergeleken. 

Het doel van dit onderzoek was om te achterhalen wat het effect is van elke compressietechniek op de grootte van de app en wat de invloed van deze techniek was op de prestaties van de app op vlak van CPU-gebruik en geheugengebruik. 

Om een antwoord te geven op de vraag wat de verschillende effecten waren van deze technieken hebben we de verschillende technieken toegepast op 3 apps met verschillende toepassingsgebieden. 1 van deze apps was een proof-of-concept app die voor dit onderzoek gecreëerd werd. 

Uit de resultaten van deze vergelijking van de technieken bleek dat de meeste technieken maar een zeer miniem effect hebben op de grootte van de APK en op de prestaties van de app. Toch zijn er ook opmerkelijke resultaten uit het onderzoek gekomen, deze komen later in deze scriptie aan bod.

Op basis van deze resultaten konden we afleiden dat deze technieken enkel bij bepaalde app-profielen een duidelijk effect zullen hebben, en dat deze technieken dan nog een averechts effect kunnen hebben op de grootte van de APK. Sommige van de aangeraden compressietechnieken worden automatisch door Android Studio toegepast en zullen dus geen extra effect hebben als deze nadien nog eens apart worden toegepast op de app.