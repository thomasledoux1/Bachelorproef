%%=============================================================================
%% Samenvatting
%%=============================================================================

%% TODO: De "abstract" of samenvatting is een kernachtige (~ 1 blz. voor een
%% thesis) synthese van het document.
%%
%% Deze aspecten moeten zeker aan bod komen:
%% - Context: waarom is dit werk belangrijk?
%% - Nood: waarom moest dit onderzocht worden?
%% - Taak: wat heb je precies gedaan?
%% - Object: wat staat in dit document geschreven?
%% - Resultaat: wat was het resultaat?
%% - Conclusie: wat is/zijn de belangrijkste conclusie(s)?
%% - Perspectief: blijven er nog vragen open die in de toekomst nog kunnen
%%    onderzocht worden? Wat is een mogelijk vervolg voor jouw onderzoek?
%%
%% LET OP! Een samenvatting is GEEN voorwoord!

%%---------- Nederlandse samenvatting -----------------------------------------
%%
%% TODO: Als je je bachelorproef in het Engels schrijft, moet je eerst een
%% Nederlandse samenvatting invoegen. Haal daarvoor onderstaande code uit
%% commentaar.
%% Wie zijn bachelorproef in het Nederlands schrijft, kan dit negeren en heel
%% deze sectie verwijderen.

\IfLanguageName{english}{%
\selectlanguage{dutch}
\chapter*{Samenvatting}
\lipsum[1-4]
\selectlanguage{english}
}{}

%%---------- Samenvatting -----------------------------------------------------
%%
%% De samenvatting in de hoofdtaal van het document

\chapter*{\IfLanguageName{dutch}{Samenvatting}{Abstract}}

De voorbije jaren hebben wearable devices meer en meer aan belang gewonnen in het dagelijks leven. Volgens onderzoek van \cite{IDC} werd het aantal verkochte wearable devices wereldwijd geschat op 20.1 miljoen, waarvan 22.9 procent Android Wear als operating system gebruikt. IDC voorspelt tegen 2020 dat er 54.6 miljoen wearable devices verkocht zullen worden, waarvan 41.8 procent Android Wear als operating system zal gebruiken. Elk noemenswaardig bedrijf heeft tegenwoordig een app laten ontwikkelen die ook op wearable devices kan gebruikt worden om zoveel mogelijk van deze mobiele gebruikers te bereiken. Hiermee gaat echter gepaard dat gebruikers meer en meer apps willen installeren op hun wearables, terwijl deze slechts over een beperkte schijfruimte beschikken. Daarom moeten de ontwikkelaars van deze mobiele applicaties ervoor zorgen dat de grootte van hun applicaties zoveel mogelijk beperkt wordt, zodat mobiele toestellen niet zonder vrije schijfruimte komen te staan. Een bijkomend probleem voor de bedrijven is dat gebruikers een beperktere set van apps zullen installeren, en er dus een grotere kans is dat sommige apps niet meer geïnstalleerd worden en er dus minder omzet wordt behaald. De bedrijven die mobiele apps laten ontwikkelen, zullen hier rekening mee moeten houden en bijvoorbeeld meer diensten in de cloud moeten aanbieden, om zo de lokaal gebruikte schijfruimte te beperken. Om dit na te gaan, zal onderzoek gedaan worden naar de applicaties van enkele van de populairste en meestgebruikte apps, zoals Spotify, Tinder en WhatsApp. In het onderzoek zal de focus gelegd worden op de gebruikte technieken bij ontwikkeling voor Android Wear. Ook zal er bij softwarebedrijf Rialto een proof-of-concept toegepast worden door bij de ontwikkeling van een kopie van de iOS-app in Android Wear. In het onderzoek zal vooral gefocust worden op de technieken die gebruikt worden in de code achter de applicaties voor bijvoorbeeld compressie van bepaalde componenten. Zo zal beschreven worden welke technieken het meest gebruikt worden en welke invloed zij hebben op de gebruikte schijfruimte. Als resultaat van dit onderzoek verwachten we te vinden dat er reeds veel technieken gebruikt worden om de gebruikte schrijfruimte door mobiele applicaties te verkleinen. Vervolgens zal een vergelijking opgemaakt worden van deze verschillende technieken. In de conclusie verwachten we te vinden dat hoewel er al goede technieken gebruikt worden, er zeker nog verbetering mogelijk is, mede doordat de shift naar wearable devices nog niet heel lang aan de gang is. Ook in de toekomst zal dit onderzoek dus zeker belangrijk blijven, aangezien het er niet op lijkt dat de groei van wearable devices snel zal stoppen. 
